\documentclass[a4paper,14pt]{article}

%%% Работа с русским языком
\usepackage{cmap}					% поиск в PDF
\usepackage{mathtext} 				% русские буквы в формулах
\usepackage[T2A]{fontenc}			% кодировка
\usepackage[utf8]{inputenc}			% кодировка исходного текста
\usepackage[english,russian]{babel}	% локализация и переносы
\usepackage{indentfirst}
\frenchspacing

\newcommand{\bz}{\mathbf{z}}
\newcommand{\bx}{\mathbf{x}}
\newcommand{\by}{\mathbf{y}}
\newcommand{\bv}{\mathbf{v}}
\newcommand{\bw}{\mathbf{w}}
\newcommand{\ba}{\mathbf{a}}
\newcommand{\bb}{\mathbf{b}}
\newcommand{\bp}{\mathbf{p}}
\newcommand{\bq}{\mathbf{q}}
\newcommand{\bt}{\mathbf{t}}
\newcommand{\bu}{\mathbf{u}}
\newcommand{\bT}{\mathbf{T}}
\newcommand{\bX}{\mathbf{X}}
\newcommand{\bZ}{\mathbf{Z}}
\newcommand{\bS}{\mathbf{S}}
\newcommand{\bH}{\mathbf{H}}
\newcommand{\bW}{\mathbf{W}}
\newcommand{\bY}{\mathbf{Y}}
\newcommand{\bU}{\mathbf{U}}
\newcommand{\bQ}{\mathbf{Q}}
\newcommand{\bP}{\mathbf{P}}
\newcommand{\bA}{\mathbf{A}}
\newcommand{\bB}{\mathbf{B}}
\newcommand{\bC}{\mathbf{C}}
\newcommand{\bE}{\mathbf{E}}
\newcommand{\bF}{\mathbf{F}}
\newcommand{\bomega}{\boldsymbol{\omega}}
\newcommand{\btheta}{\boldsymbol{\theta}}
\newcommand{\bgamma}{\boldsymbol{\gamma}}
\newcommand{\bdelta}{\boldsymbol{\delta}}
\newcommand{\bPsi}{\boldsymbol{\Psi}}
\newcommand{\bpsi}{\boldsymbol{\psi}}
\newcommand{\bxi}{\boldsymbol{\xi}}
\newcommand{\bchi}{\boldsymbol{\chi}}
\newcommand{\bzeta}{\boldsymbol{\zeta}}
\newcommand{\blambda}{\boldsymbol{\lambda}}
\newcommand{\beps}{\boldsymbol{\varepsilon}}
\newcommand{\bZeta}{\boldsymbol{Z}}
% mathcal
\newcommand{\cX}{\mathcal{X}}
\newcommand{\cY}{\mathcal{Y}}
\newcommand{\cW}{\mathcal{W}}

\newcommand{\dH}{\mathbb{H}}
\newcommand{\dR}{\mathbb{R}}
% transpose
\newcommand{\T}{^{\mathsf{T}}}

\renewcommand{\epsilon}{\ensuremath{\varepsilon}}
\renewcommand{\phi}{\ensuremath{\varphi}}
\renewcommand{\kappa}{\ensuremath{\varkappa}}
\renewcommand{\le}{\ensuremath{\leqslant}}
\renewcommand{\leq}{\ensuremath{\leqslant}}
\renewcommand{\ge}{\ensuremath{\geqslant}}
\renewcommand{\geq}{\ensuremath{\geqslant}}
\renewcommand{\emptyset}{\varnothing}

%%% Дополнительная работа с математикой
\usepackage{amsmath,amsfonts,amssymb,amsthm,mathtools} % AMS
\usepackage{icomma} % "Умная" запятая: $0,2$ --- число, $0, 2$ --- перечисление

%% Номера формул
%\mathtoolsset{showonlyrefs=true} % Показывать номера только у тех формул, на которые есть \eqref{} в тексте.
%\usepackage{leqno} % Нумереация формул слева

%% Свои команды
\DeclareMathOperator{\sgn}{\mathop{sgn}}

%% Перенос знаков в формулах (по Львовскому)
\newcommand*{\hm}[1]{#1\nobreak\discretionary{}
	{\hbox{$\mathsurround=0pt #1$}}{}}

%%% Работа с картинками
\usepackage{graphicx}  % Для вставки рисунков
\setlength\fboxsep{3pt} % Отступ рамки \fbox{} от рисунка
\setlength\fboxrule{1pt} % Толщина линий рамки \fbox{}
\usepackage{wrapfig} % Обтекание рисунков текстом

%%% Работа с таблицами
\usepackage{array,tabularx,tabulary,booktabs} % Дополнительная работа с таблицами
\usepackage{longtable}  % Длинные таблицы
\usepackage{multirow} % Слияние строк в таблице

%%% Теоремы
\theoremstyle{plain} % Это стиль по умолчанию, его можно не переопределять.
\newtheorem{theorem}{Теорема}[section]
\newtheorem{proposition}[theorem]{Утверждение}

\theoremstyle{definition} % "Определение"
\newtheorem{corollary}{Следствие}[theorem]
\newtheorem{problem}{Задача}[section]

\theoremstyle{remark} % "Примечание"
\newtheorem*{nonum}{Решение}

%%% Программирование
\usepackage{etoolbox} % логические операторы

%%% Страница
\usepackage{extsizes} % Возможность сделать 14-й шрифт
\usepackage{geometry} % Простой способ задавать поля
\geometry{top=25mm}
\geometry{bottom=35mm}
\geometry{left=35mm}
\geometry{right=20mm}
%
%\usepackage{fancyhdr} % Колонтитулы
% 	\pagestyle{fancy}
%\renewcommand{\headrulewidth}{0pt}  % Толщина линейки, отчеркивающей верхний колонтитул
% 	\lfoot{Нижний левый}
% 	\rfoot{Нижний правый}
% 	\rhead{Верхний правый}
% 	\chead{Верхний в центре}
% 	\lhead{Верхний левый}
%	\cfoot{Нижний в центре} % По умолчанию здесь номер страницы

\usepackage{setspace} % Интерлиньяж
%\onehalfspacing % Интерлиньяж 1.5
%\doublespacing % Интерлиньяж 2
%\singlespacing % Интерлиньяж 1

\usepackage{lastpage} % Узнать, сколько всего страниц в документе.

\usepackage{soul} % Модификаторы начертания

\usepackage{hyperref}
\usepackage[usenames,dvipsnames,svgnames,table,rgb]{xcolor}
\hypersetup{				% Гиперссылки
	unicode=true,           % русские буквы в раздела PDF
	pdftitle={Заголовок},   % Заголовок
	pdfauthor={Автор},      % Автор
	pdfsubject={Тема},      % Тема
	pdfcreator={Создатель}, % Создатель
	pdfproducer={Производитель}, % Производитель
	pdfkeywords={keyword1} {key2} {key3}, % Ключевые слова
	colorlinks=true,       	% false: ссылки в рамках; true: цветные ссылки
	linkcolor=red,          % внутренние ссылки
	citecolor=black,        % на библиографию
	filecolor=magenta,      % на файлы
	urlcolor=cyan           % на URL
}

\usepackage{csquotes} % Еще инструменты для ссылок

% \usepackage[style=authoryear,maxcitenames=2,backend=biber,sorting=nty]{biblatex}
% \addbibresource{references.bib}
\usepackage[numbers]{natbib}

\usepackage{multicol} % Несколько колонок

\usepackage{tikz} % Работа с графикой
\usepackage{pgfplots}
\usepackage{pgfplotstable}


\author{Eduard Vladimirov, Daniil Kazachkov, Vadim Strijov}
\title{\textbf{Cross-attention and CCA in EEG}}
\date{\today}

\begin{document}
	\maketitle
	
	\section*{Abstract}
        На сегодняшний день работа с мультимодальными данными набирает всё большую популярность: учет взаимосвязей между ними улучшает качество предсказания. В этой статье мы предлагаем новую архитектуру, использующую преимущества алгоритма Canonical Correlation Analysis (CCA) и механизма Attention. Ниже будет показано, что CCA - частный случай Attention, а значит, мультимодальность можно встроить внутрь фреймворка Attention. Работа полученной модели иллюстрируется на задаче классификации удара теннисного мяча по датасету Real World Table Tennis. 

        $\mathbf{keywords:}$ CCA, Attention, BCI, online-classification.

        \section*{Introduction}

        Мультимодальность - мощный инструмент для улучшения качества ответов модели [What Makes Multi-modal Learning Better than Single]. Канонический корреляционный анализ (CCA) [A Survey on Canonical Correlation Analysis] является очень популярным статистическим методом, снижения размерности двух множеств данных, при котором корреляция между парными переменными в общем подпространстве взаимно максимизируется. В таких работах как [Correlational Neural Networks], [A survey on deep multimodal learning for computer vision: advances, trends, applications, and datasets] авторы показали, что он улучшает качество в задачах сопоставления событий. Однако CCA может моделировать лишь линейные зависимости. \\
        В противоположность этому, механизм Attention находит сложные, нелинейные зависимости. Усовершенствование cross-attention позволит лучше отсеивать информацию, снизит размерность пространства и тем самым ускорит вычисления. \\
        Ставя перед собой цель использовать преимущества каждого метода, мы представляем модель CCAT: Canonical-Correlation Attention Transformer. (тут надо описать общую концепцию, но я пока не придумал как).
        
        \vspace{10mm}
        
        В качестве датасета для нашей задачи мы взяли "Real World Table Tennis" \cite{amanda2024dataset}, из которого оставили уже предобработанные данные акселерометра и ЭЭГ. С помощью CCA мы перешли в риманово латентное пространство меньшей размерности, где 

        \subsection*{Related Works}

        Нейрокомпьютерный интерфейс (Brain-Computer Interface, BCI) \cite{shih2012brain} считывает сигналы поверхности кортекса головного мозга, анализирует и переводит в команды исполняющей системы. Результатом измерений является временной ряд напряжений на электродах, который используется в задаче декодирования сигнала. \\

        
        \section*{CCAT: Canonical-Correlation Attention Transformer}
        \subsection{Model}
        Мы продолжаем расширять область применимости CCA и представляем способ встраивания его в Attention.
        
	\nocite{*}
        \bibliographystyle{unsrt} % Use a numbered bibliography style
        \bibliography{references.bib}
	% \printbibliography
	
\end{document}
